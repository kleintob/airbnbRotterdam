% Options for packages loaded elsewhere
\PassOptionsToPackage{unicode}{hyperref}
\PassOptionsToPackage{hyphens}{url}
%
\documentclass[
]{article}
\usepackage{amsmath,amssymb}
\usepackage{iftex}
\ifPDFTeX
  \usepackage[T1]{fontenc}
  \usepackage[utf8]{inputenc}
  \usepackage{textcomp} % provide euro and other symbols
\else % if luatex or xetex
  \usepackage{unicode-math} % this also loads fontspec
  \defaultfontfeatures{Scale=MatchLowercase}
  \defaultfontfeatures[\rmfamily]{Ligatures=TeX,Scale=1}
\fi
\usepackage{lmodern}
\ifPDFTeX\else
  % xetex/luatex font selection
\fi
% Use upquote if available, for straight quotes in verbatim environments
\IfFileExists{upquote.sty}{\usepackage{upquote}}{}
\IfFileExists{microtype.sty}{% use microtype if available
  \usepackage[]{microtype}
  \UseMicrotypeSet[protrusion]{basicmath} % disable protrusion for tt fonts
}{}
\makeatletter
\@ifundefined{KOMAClassName}{% if non-KOMA class
  \IfFileExists{parskip.sty}{%
    \usepackage{parskip}
  }{% else
    \setlength{\parindent}{0pt}
    \setlength{\parskip}{6pt plus 2pt minus 1pt}}
}{% if KOMA class
  \KOMAoptions{parskip=half}}
\makeatother
\usepackage{xcolor}
\usepackage[margin=1in]{geometry}
\usepackage{color}
\usepackage{fancyvrb}
\newcommand{\VerbBar}{|}
\newcommand{\VERB}{\Verb[commandchars=\\\{\}]}
\DefineVerbatimEnvironment{Highlighting}{Verbatim}{commandchars=\\\{\}}
% Add ',fontsize=\small' for more characters per line
\usepackage{framed}
\definecolor{shadecolor}{RGB}{248,248,248}
\newenvironment{Shaded}{\begin{snugshade}}{\end{snugshade}}
\newcommand{\AlertTok}[1]{\textcolor[rgb]{0.94,0.16,0.16}{#1}}
\newcommand{\AnnotationTok}[1]{\textcolor[rgb]{0.56,0.35,0.01}{\textbf{\textit{#1}}}}
\newcommand{\AttributeTok}[1]{\textcolor[rgb]{0.13,0.29,0.53}{#1}}
\newcommand{\BaseNTok}[1]{\textcolor[rgb]{0.00,0.00,0.81}{#1}}
\newcommand{\BuiltInTok}[1]{#1}
\newcommand{\CharTok}[1]{\textcolor[rgb]{0.31,0.60,0.02}{#1}}
\newcommand{\CommentTok}[1]{\textcolor[rgb]{0.56,0.35,0.01}{\textit{#1}}}
\newcommand{\CommentVarTok}[1]{\textcolor[rgb]{0.56,0.35,0.01}{\textbf{\textit{#1}}}}
\newcommand{\ConstantTok}[1]{\textcolor[rgb]{0.56,0.35,0.01}{#1}}
\newcommand{\ControlFlowTok}[1]{\textcolor[rgb]{0.13,0.29,0.53}{\textbf{#1}}}
\newcommand{\DataTypeTok}[1]{\textcolor[rgb]{0.13,0.29,0.53}{#1}}
\newcommand{\DecValTok}[1]{\textcolor[rgb]{0.00,0.00,0.81}{#1}}
\newcommand{\DocumentationTok}[1]{\textcolor[rgb]{0.56,0.35,0.01}{\textbf{\textit{#1}}}}
\newcommand{\ErrorTok}[1]{\textcolor[rgb]{0.64,0.00,0.00}{\textbf{#1}}}
\newcommand{\ExtensionTok}[1]{#1}
\newcommand{\FloatTok}[1]{\textcolor[rgb]{0.00,0.00,0.81}{#1}}
\newcommand{\FunctionTok}[1]{\textcolor[rgb]{0.13,0.29,0.53}{\textbf{#1}}}
\newcommand{\ImportTok}[1]{#1}
\newcommand{\InformationTok}[1]{\textcolor[rgb]{0.56,0.35,0.01}{\textbf{\textit{#1}}}}
\newcommand{\KeywordTok}[1]{\textcolor[rgb]{0.13,0.29,0.53}{\textbf{#1}}}
\newcommand{\NormalTok}[1]{#1}
\newcommand{\OperatorTok}[1]{\textcolor[rgb]{0.81,0.36,0.00}{\textbf{#1}}}
\newcommand{\OtherTok}[1]{\textcolor[rgb]{0.56,0.35,0.01}{#1}}
\newcommand{\PreprocessorTok}[1]{\textcolor[rgb]{0.56,0.35,0.01}{\textit{#1}}}
\newcommand{\RegionMarkerTok}[1]{#1}
\newcommand{\SpecialCharTok}[1]{\textcolor[rgb]{0.81,0.36,0.00}{\textbf{#1}}}
\newcommand{\SpecialStringTok}[1]{\textcolor[rgb]{0.31,0.60,0.02}{#1}}
\newcommand{\StringTok}[1]{\textcolor[rgb]{0.31,0.60,0.02}{#1}}
\newcommand{\VariableTok}[1]{\textcolor[rgb]{0.00,0.00,0.00}{#1}}
\newcommand{\VerbatimStringTok}[1]{\textcolor[rgb]{0.31,0.60,0.02}{#1}}
\newcommand{\WarningTok}[1]{\textcolor[rgb]{0.56,0.35,0.01}{\textbf{\textit{#1}}}}
\usepackage{graphicx}
\makeatletter
\def\maxwidth{\ifdim\Gin@nat@width>\linewidth\linewidth\else\Gin@nat@width\fi}
\def\maxheight{\ifdim\Gin@nat@height>\textheight\textheight\else\Gin@nat@height\fi}
\makeatother
% Scale images if necessary, so that they will not overflow the page
% margins by default, and it is still possible to overwrite the defaults
% using explicit options in \includegraphics[width, height, ...]{}
\setkeys{Gin}{width=\maxwidth,height=\maxheight,keepaspectratio}
% Set default figure placement to htbp
\makeatletter
\def\fps@figure{htbp}
\makeatother
\setlength{\emergencystretch}{3em} % prevent overfull lines
\providecommand{\tightlist}{%
  \setlength{\itemsep}{0pt}\setlength{\parskip}{0pt}}
\setcounter{secnumdepth}{-\maxdimen} % remove section numbering
\ifLuaTeX
  \usepackage{selnolig}  % disable illegal ligatures
\fi
\usepackage{bookmark}
\IfFileExists{xurl.sty}{\usepackage{xurl}}{} % add URL line breaks if available
\urlstyle{same}
\hypersetup{
  pdftitle={Chapter 8: heteroskedasticity},
  hidelinks,
  pdfcreator={LaTeX via pandoc}}

\title{Chapter 8: heteroskedasticity}
\author{}
\date{\vspace{-2.5em}}

\begin{document}
\maketitle

We start again by loading the data and filtering it so that we restrict
it to properties that host at most 6 people. Also, as in Chapter 3, we
again create a variable \texttt{review\_scores\_rating\_standardized}
with the standardized review score.

\begin{Shaded}
\begin{Highlighting}[]
\CommentTok{\# Load the datasets from the RData file}
\FunctionTok{load}\NormalTok{(}\StringTok{"../dataCreated/listings\_clean.RData"}\NormalTok{)}

\NormalTok{listings\_clean\_filtered }\OtherTok{\textless{}{-}}\NormalTok{ listings\_clean }\SpecialCharTok{\%\textgreater{}\%}
  \FunctionTok{filter}\NormalTok{(accommodates }\SpecialCharTok{\textless{}=} \DecValTok{6}\NormalTok{) }

\CommentTok{\# Standardize review\_scores\_rating}
\NormalTok{listings\_clean\_filtered }\OtherTok{\textless{}{-}}\NormalTok{ listings\_clean\_filtered }\SpecialCharTok{\%\textgreater{}\%}
  \FunctionTok{mutate}\NormalTok{(}\AttributeTok{review\_scores\_rating\_standardized =} 
\NormalTok{           (review\_scores\_rating }\SpecialCharTok{{-}} \FunctionTok{mean}\NormalTok{(review\_scores\_rating, }\AttributeTok{na.rm =} \ConstantTok{TRUE}\NormalTok{)) }\SpecialCharTok{/} 
           \FunctionTok{sd}\NormalTok{(review\_scores\_rating, }\AttributeTok{na.rm =} \ConstantTok{TRUE}\NormalTok{))}
\end{Highlighting}
\end{Shaded}

Our point of departure for Chapter 8 is, as for Chapter 4, the richer
model from Chapter 3 where we regress the log price on
\texttt{review\_scores\_rating}, \texttt{accommodates}, and 4
neighborhood characteristics.

\begin{Shaded}
\begin{Highlighting}[]
\NormalTok{estimatesFullModel }\OtherTok{\textless{}{-}} \FunctionTok{lm}\NormalTok{(}\FunctionTok{log}\NormalTok{(price) }\SpecialCharTok{\textasciitilde{}}\NormalTok{ review\_scores\_rating\_standardized }\SpecialCharTok{+}\NormalTok{ accommodates }\SpecialCharTok{+}\NormalTok{ Centrality }\SpecialCharTok{+}\NormalTok{ Quietness }\SpecialCharTok{+}\NormalTok{ Coolness }\SpecialCharTok{+}\NormalTok{ Fanciness, }\AttributeTok{data =}\NormalTok{ listings\_clean\_filtered)}
\FunctionTok{summary}\NormalTok{(estimatesFullModel)}
\end{Highlighting}
\end{Shaded}

\begin{verbatim}
## 
## Call:
## lm(formula = log(price) ~ review_scores_rating_standardized + 
##     accommodates + Centrality + Quietness + Coolness + Fanciness, 
##     data = listings_clean_filtered)
## 
## Residuals:
##      Min       1Q   Median       3Q      Max 
## -1.48845 -0.24997  0.01026  0.23530  0.94151 
## 
## Coefficients:
##                                   Estimate Std. Error t value Pr(>|t|)    
## (Intercept)                        3.10015    0.49275   6.292    8e-10 ***
## review_scores_rating_standardized  0.05000    0.01887   2.650  0.00837 ** 
## accommodates                       0.21641    0.01480  14.625  < 2e-16 ***
## Centrality                         0.08591    0.04406   1.950  0.05186 .  
## Quietness                          0.12896    0.04451   2.897  0.00397 ** 
## Coolness                           0.09606    0.07834   1.226  0.22079    
## Fanciness                         -0.14765    0.05382  -2.744  0.00634 ** 
## ---
## Signif. codes:  0 '***' 0.001 '**' 0.01 '*' 0.05 '.' 0.1 ' ' 1
## 
## Residual standard error: 0.3834 on 414 degrees of freedom
## Multiple R-squared:  0.3494, Adjusted R-squared:   0.34 
## F-statistic: 37.05 on 6 and 414 DF,  p-value: < 2.2e-16
\end{verbatim}

The reported point estimates are unbiased (Chapter 3) and consistent
(Chapter 5) under MLR1-MLR4. The standard errors are valid when we make
the additional assumption of homoskedasticity.

Next, we test the null of homoskedasticity.

\begin{Shaded}
\begin{Highlighting}[]
\FunctionTok{bptest}\NormalTok{(estimatesFullModel)}
\end{Highlighting}
\end{Shaded}

\begin{verbatim}
## 
##  studentized Breusch-Pagan test
## 
## data:  estimatesFullModel
## BP = 19.232, df = 6, p-value = 0.00379
\end{verbatim}

The null of homoskedasticity is rejected.

We can directly look at results with robust standard errors.

\begin{Shaded}
\begin{Highlighting}[]
\FunctionTok{coeftest}\NormalTok{(estimatesFullModel, }\AttributeTok{vcov=}\NormalTok{hccm)}
\end{Highlighting}
\end{Shaded}

\begin{verbatim}
## 
## t test of coefficients:
## 
##                                    Estimate Std. Error t value  Pr(>|t|)    
## (Intercept)                        3.100154   0.495124  6.2614 9.558e-10 ***
## review_scores_rating_standardized  0.049999   0.019180  2.6067  0.009471 ** 
## accommodates                       0.216411   0.015438 14.0180 < 2.2e-16 ***
## Centrality                         0.085908   0.042210  2.0353  0.042462 *  
## Quietness                          0.128962   0.050107  2.5737  0.010408 *  
## Coolness                           0.096064   0.076005  1.2639  0.206973    
## Fanciness                         -0.147651   0.058117 -2.5406  0.011432 *  
## ---
## Signif. codes:  0 '***' 0.001 '**' 0.01 '*' 0.05 '.' 0.1 ' ' 1
\end{verbatim}

Some are smaller, some are bigger.

Next, we do weighted least squares. First, we need to estimate the
residuals from the full model.

\begin{Shaded}
\begin{Highlighting}[]
\CommentTok{\# Obtain residuals}
\NormalTok{residuals\_full\_model }\OtherTok{\textless{}{-}} \FunctionTok{residuals}\NormalTok{(estimatesFullModel)}
\end{Highlighting}
\end{Shaded}

Then, we specify that \[
\begin{align*}
\text{var}(u \mid \text{review_scores_rating_standardized}, \text{accommodates}, \text{Centrality}, \text{Quietness}, \text{Coolness}, \text{Fanciness}) \\
= \sigma^2 \cdot exp(\delta_0 + \delta_1 \text{review_scores_rating_standardized} + \delta_2 \text{accommodates} \\ + \delta_3 \text{Centrality} + \delta_4 \text{Quietness} + \delta_5 \text{Coolness} + \delta_5 \text{Fanciness}).
\end{align*}
\] This means that we can estimate the parameters
\(\delta_0, \ldots, \delta_5\) by regressing the log of the squared
residuals on the explanatory variables (see slides 21 and 23): \[
\begin{align*}
\text{log}(\hat u_i^2) = \tilde \delta_0 + \delta_1 \text{review_scores_rating_standardized} + \delta_2 \text{accommodates} \\ + \delta_3 \text{Centrality} + \delta_4 \text{Quietness} + \delta_5 \text{Coolness} + \delta_5 \text{Fanciness} + w,
\end{align*}
\] where \(\tilde \delta_0\) is equal to \(\log(\sigma^2) + \delta_0\).

\begin{Shaded}
\begin{Highlighting}[]
\CommentTok{\# Step 1 was to run the regression above}
\CommentTok{\# Step 2: Regress squared residuals on predictors to model heteroskedasticity}
\NormalTok{log\_squared\_residuals }\OtherTok{\textless{}{-}} \FunctionTok{log}\NormalTok{(residuals\_full\_model}\SpecialCharTok{\^{}}\DecValTok{2}\NormalTok{)}
\NormalTok{model\_resid\_squared }\OtherTok{\textless{}{-}} \FunctionTok{lm}\NormalTok{(log\_squared\_residuals }\SpecialCharTok{\textasciitilde{}}\NormalTok{ review\_scores\_rating\_standardized }\SpecialCharTok{+}\NormalTok{ accommodates }
                          \SpecialCharTok{+}\NormalTok{ Centrality }\SpecialCharTok{+}\NormalTok{ Quietness }\SpecialCharTok{+}\NormalTok{ Coolness }\SpecialCharTok{+}\NormalTok{ Fanciness, }\AttributeTok{data =}\NormalTok{ listings\_clean\_filtered)}
\end{Highlighting}
\end{Shaded}

Next, we calculate the weights as the inverse of the square root of the
fitted values from the regression above. We have to pay attention to
also un-do taking the log of the squared residuals.\footnote{See also
  Section 6-4c in the Wooldrige book on p.~205ff on predicting the level
  of the dependent variable when one runs a regression that has it in
  logs on the left hand side. This raises an issue related to the
  constant term that we can however ignore here. In brief, things
  related to the smearing factor will only affect the constant term,
  which is not important here.}

\begin{Shaded}
\begin{Highlighting}[]
\CommentTok{\# Obtain fitted values from this regression}
\NormalTok{fitted\_values }\OtherTok{\textless{}{-}} \FunctionTok{exp}\NormalTok{(}\FunctionTok{fitted}\NormalTok{(model\_resid\_squared))}

\CommentTok{\# Step 3: Calculate weights as inverse of the square root of fitted values}
\NormalTok{weights }\OtherTok{\textless{}{-}} \DecValTok{1} \SpecialCharTok{/} \FunctionTok{sqrt}\NormalTok{(fitted\_values)}
\end{Highlighting}
\end{Shaded}

Finally, we run the WLS regression with these weights.

\begin{Shaded}
\begin{Highlighting}[]
\CommentTok{\# Step 4: Run the WLS regression with these weights}
\NormalTok{wls\_model }\OtherTok{\textless{}{-}} \FunctionTok{lm}\NormalTok{(}\FunctionTok{log}\NormalTok{(price) }\SpecialCharTok{\textasciitilde{}}\NormalTok{ review\_scores\_rating\_standardized }\SpecialCharTok{+}\NormalTok{ accommodates }
                \SpecialCharTok{+}\NormalTok{ Centrality }\SpecialCharTok{+}\NormalTok{ Quietness }\SpecialCharTok{+}\NormalTok{ Coolness }\SpecialCharTok{+}\NormalTok{ Fanciness, }
                \AttributeTok{data =}\NormalTok{ listings\_clean\_filtered, }\AttributeTok{weights =}\NormalTok{ weights)}

\CommentTok{\# Summary of the WLS model}
\FunctionTok{summary}\NormalTok{(wls\_model)}
\end{Highlighting}
\end{Shaded}

\begin{verbatim}
## 
## Call:
## lm(formula = log(price) ~ review_scores_rating_standardized + 
##     accommodates + Centrality + Quietness + Coolness + Fanciness, 
##     data = listings_clean_filtered, weights = weights)
## 
## Weighted Residuals:
##     Min      1Q  Median      3Q     Max 
## -4.1537 -0.5811  0.0196  0.5532  2.1320 
## 
## Coefficients:
##                                   Estimate Std. Error t value Pr(>|t|)    
## (Intercept)                        3.15303    0.49368   6.387 4.56e-10 ***
## review_scores_rating_standardized  0.05891    0.01847   3.189  0.00154 ** 
## accommodates                       0.20300    0.01362  14.901  < 2e-16 ***
## Centrality                         0.07760    0.04441   1.747  0.08129 .  
## Quietness                          0.12370    0.04616   2.680  0.00766 ** 
## Coolness                           0.09661    0.08038   1.202  0.23009    
## Fanciness                         -0.13630    0.05086  -2.680  0.00766 ** 
## ---
## Signif. codes:  0 '***' 0.001 '**' 0.01 '*' 0.05 '.' 0.1 ' ' 1
## 
## Residual standard error: 0.8678 on 414 degrees of freedom
## Multiple R-squared:  0.361,  Adjusted R-squared:  0.3518 
## F-statistic: 38.99 on 6 and 414 DF,  p-value: < 2.2e-16
\end{verbatim}

Finally, we use the stargazer package to show OLS results side-by-side
OLS results with robust standard errors and weighted least squares
results without and with robust standard errors. For the WLS estimates,
in theory, one does not need them if one gets the weighting function
right.

\begin{Shaded}
\begin{Highlighting}[]
\CommentTok{\# Calculate robust standard errors for both OLS and WLS models}
\NormalTok{robust\_se\_ols }\OtherTok{\textless{}{-}} \FunctionTok{sqrt}\NormalTok{(}\FunctionTok{diag}\NormalTok{(}\FunctionTok{hccm}\NormalTok{(estimatesFullModel, }\AttributeTok{type =} \StringTok{"hc3"}\NormalTok{)))}
\NormalTok{robust\_se\_wls }\OtherTok{\textless{}{-}} \FunctionTok{sqrt}\NormalTok{(}\FunctionTok{diag}\NormalTok{(}\FunctionTok{hccm}\NormalTok{(wls\_model, }\AttributeTok{type =} \StringTok{"hc3"}\NormalTok{)))}

\CommentTok{\# Compare OLS with standard errors, OLS with robust standard errors, WLS, and WLS with robust SE}
\FunctionTok{stargazer}\NormalTok{(estimatesFullModel, estimatesFullModel, wls\_model, wls\_model,}
          \AttributeTok{se =} \FunctionTok{list}\NormalTok{(}\ConstantTok{NULL}\NormalTok{, robust\_se\_ols, }\ConstantTok{NULL}\NormalTok{, robust\_se\_wls),}
          \AttributeTok{column.labels =} \FunctionTok{c}\NormalTok{(}\StringTok{"OLS"}\NormalTok{, }\StringTok{"OLS (Robust SE)"}\NormalTok{, }\StringTok{"WLS"}\NormalTok{, }\StringTok{"WLS (Robust SE)"}\NormalTok{),}
          \AttributeTok{column.separate =} \FunctionTok{c}\NormalTok{(}\DecValTok{1}\NormalTok{, }\DecValTok{1}\NormalTok{, }\DecValTok{1}\NormalTok{, }\DecValTok{1}\NormalTok{),}
          \AttributeTok{type =} \StringTok{"text"}\NormalTok{, }\AttributeTok{keep.stat =} \FunctionTok{c}\NormalTok{(}\StringTok{"n"}\NormalTok{, }\StringTok{"rsq"}\NormalTok{))}
\end{Highlighting}
\end{Shaded}

\begin{verbatim}
## 
## =====================================================================================
##                                                   Dependent variable:                
##                                   ---------------------------------------------------
##                                                       log(price)                     
##                                      OLS    OLS (Robust SE)    WLS    WLS (Robust SE)
##                                      (1)          (2)          (3)          (4)      
## -------------------------------------------------------------------------------------
## review_scores_rating_standardized 0.050***     0.050***     0.059***     0.059***    
##                                    (0.019)      (0.019)      (0.018)      (0.017)    
##                                                                                      
## accommodates                      0.216***     0.216***     0.203***     0.203***    
##                                    (0.015)      (0.015)      (0.014)      (0.015)    
##                                                                                      
## Centrality                         0.086*       0.086**      0.078*       0.078**    
##                                    (0.044)      (0.042)      (0.044)      (0.039)    
##                                                                                      
## Quietness                         0.129***      0.129**     0.124***      0.124**    
##                                    (0.045)      (0.050)      (0.046)      (0.049)    
##                                                                                      
## Coolness                            0.096        0.096        0.097        0.097     
##                                    (0.078)      (0.076)      (0.080)      (0.075)    
##                                                                                      
## Fanciness                         -0.148***    -0.148**     -0.136***    -0.136**    
##                                    (0.054)      (0.058)      (0.051)      (0.057)    
##                                                                                      
## Constant                          3.100***     3.100***     3.153***     3.153***    
##                                    (0.493)      (0.495)      (0.494)      (0.485)    
##                                                                                      
## -------------------------------------------------------------------------------------
## Observations                         421          421          421          421      
## R2                                  0.349        0.349        0.361        0.361     
## =====================================================================================
## Note:                                                     *p<0.1; **p<0.05; ***p<0.01
\end{verbatim}

Comparing column (1) and (2) shows that standard errors are almost
unchanged. They sometimes even get smaller. This can in principle
happen, as it does here. Usually, they get bigger. For the WLS
estimates, standard errors are bigger when they are robust, but
tentatively smaller than the robust OLS ones.

\end{document}
